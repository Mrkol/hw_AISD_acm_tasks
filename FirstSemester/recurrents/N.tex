\documentclass[11pt]{article}

\usepackage[utf8]{inputenc}
\usepackage[russian]{babel}
\usepackage{listings}

\usepackage{geometry}
\geometry{a4paper}
\lstset{tabsize=2}

\title{Рекуренты-4}
\author{Санду Р.А.}

\begin{document}
\maketitle

\[
	T\left(n\right)=\sqrt{n}T\left(\left\lceil\sqrt{n}\right\rceil\right)+n
\]

Для простоты будем оценвивать функцию

\[
	T\left(n\right)=\sqrt{n}T\left(\sqrt{n}\right)+n
\]

Раскрутим рекуренту 3 раза:

\[
	T\left(n\right) = \sqrt{n}\sqrt[4]{n}T\left(\sqrt[4]{n}\right)+2n
\]

\[
	T\left(n\right) = \sqrt{n}\sqrt[4]{n}\sqrt[8]{n}T\left(\sqrt[8]{n}\right)+3n
\]

\[
	T\left(n\right) = \sqrt{n}\sqrt[4]{n}\sqrt[8]{n}\sqrt[16]{n}T\left(\sqrt[16]{n}\right)+4n
\]

Очевидна закономерность: на каждой итерации коэффициент перед нерекурсивным членом увеличивается на 1. Возникает вопрос: сколько итераций совершит алгоритм? Ответом будет наименьшее решение следующего неравенства относительно x:

\[
	n^{2^{-x}} \leq 2
\]

(берём 2, т.к. итерированный корень из действительного числа больше 1 никогда не достигает 1, а время работы $T(2)$ можно считать константой)

\[
	\frac{\log_2{n}}{2^x} \leq \log_2{2}
\]

\[
	\log_2{n} \leq 2^x
\]

\[
	x \geq \log_2{\log_2{n}}
\]

Итого ассимптотика будет

\[
	T\left(n\right) = \Theta\left(nx\right) = \Theta\left(n\log_2{\log_2{n}}\right)
\]

\end{document}