\documentclass[11pt]{article}

\usepackage[utf8]{inputenc}
\usepackage[russian]{babel}
\usepackage{listings}

\usepackage{geometry}
\usepackage{amsmath}
\geometry{a4paper}
\lstset{tabsize=2}

\title{Рекуренты-3}
\author{Санду Р.А.}

\begin{document}
\maketitle

\begin{equation}
	T\left(n\right)=2T\left(\left\lfloor\frac{n}{2}\right\rfloor\right)+n\log_2{n}
\end{equation}

Легко заметить, что ответ — $\Theta\left(n\log^2_2{n}\right)$

Докажем это по индукции. Верхняя оценка:

\begin{equation}
	T\left(n\right) \leq Cn\log^2_2{n}
\end{equation}

Переход:

\begin{equation}
	T\left(\left\lfloor\frac{n}{2}\right\rfloor\right) 
	\leq 
	C\left\lfloor\frac{n}{2}\right\rfloor \log^2_2{\left\lfloor\frac{n}{2}\right\rfloor}
\end{equation}
Целую часть оценим сверху самим числом:
\begin{equation}
	2T\left(\left\lfloor\frac{n}{2}\right\rfloor\right) 
	\leq 
	C n\left(\log_2{n}-1\right)^2
\end{equation}

\begin{equation}
	2T\left(\left\lfloor\frac{n}{2}\right\rfloor\right) 
	\leq 
	Cn\log^2_2{n} - 2Cn\log_2{n} + nC
\end{equation}
Так как для достаточно больших $n$ имеем $n - n\log_2{n} < 0$, ослабим оценку:
\begin{equation}
	2T\left(\left\lfloor\frac{n}{2}\right\rfloor\right) 
	\leq 
	Cn\log^2_2{n} - Cn\log_2{n}
\end{equation}
\begin{equation}
	2T\left(\left\lfloor\frac{n}{2}\right\rfloor\right) + Cn\log_2{n}
	\leq 
	Cn\log^2_2{n}
\end{equation}
Но тогда для достаточно больших $C$ получаем
\begin{equation}
	T\left(n\right)
	=
	2T\left(\left\lfloor\frac{n}{2}\right\rfloor\right) + n\log_2{n}
	\leq
	2T\left(\left\lfloor\frac{n}{2}\right\rfloor\right) + Cn\log_2{n} 
	\leq 
	Cn\log^2_2{n}
\end{equation}

\begin{equation}
	T\left(n\right)
	\leq 
	Cn\log^2_2{n}
\end{equation}

Что и требовалось доказать. За базу возьмём $n=2$ и $n=3$, для них очевидно выполняется, константа подбирается. Тогда переход индукции доказывает утверждение для всех б\'{о}льших $n$.

\pagebreak

Нижняя оценка:

\begin{equation}
	T\left(n\right) \geq  cn\log^2_2{n}
\end{equation}
Переход:
\begin{equation}
	T\left(\left\lfloor\frac{n}{2}\right\rfloor\right) \geq  c\left\lfloor\frac{n}{2}\right\rfloor\log^2_2{\left\lfloor\frac{n}{2}\right\rfloor}
\end{equation}
Оценим целую часть снизу:
\begin{equation}
	T\left(\left\lfloor\frac{n}{2}\right\rfloor\right) \geq  c \frac{n - 1}{2} \log^2_2{\frac{n}{4}}
\end{equation}
Преобразуем:
\begin{equation}
	2T\left(\left\lfloor\frac{n}{2}\right\rfloor\right) \geq  c \left(n - 1\right) \left(\log_2{n}-2\right)^2
\end{equation}

\begin{equation}
	2T\left(\left\lfloor\frac{n}{2}\right\rfloor\right) \geq  c \left(n - 1\right) \left(\log^2_2{n}-4\log_2{n}+4\right)
\end{equation}

\begin{equation}
	2T\left(\left\lfloor\frac{n}{2}\right\rfloor\right) \geq  cn\log^2_2{n}-4cn\log_2{n}+\left(4cn - c\log^2_2{n}+4c\log_2{n}-4c\right)
\end{equation}
Для достаточно больших $n$ можно ослабить:
\begin{equation}
	2T\left(\left\lfloor\frac{n}{2}\right\rfloor\right)
	\geq
	cn\log^2_2{n} - 4cn\log_2{n}
\end{equation}

\begin{equation}
	2T\left(\left\lfloor\frac{n}{2}\right\rfloor\right) + n\log_2{n}
	\geq
	cn\log^2_2{n} + n\log_2{n} - 4cn\log_2{n}
\end{equation}

Для достаточно малых $c$:

\begin{equation}
	2T\left(\left\lfloor\frac{n}{2}\right\rfloor\right) + n\log_2{n}
	\geq
	cn\log^2_2{n}
\end{equation}

\begin{equation}
	T\left(n\right)
	\geq
	cn\log^2_2{n}
\end{equation}

Что и требовалось. База берётся аналогично.

Таким образом, действительно, $T(n) = \Theta(n\log^2_2{n})$.



\end{document}