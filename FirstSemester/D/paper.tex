\documentclass[11pt]{article}

\usepackage[utf8]{inputenc}
\usepackage[russian]{babel}
\usepackage{listings}

\usepackage{geometry}
\geometry{a4paper}
\lstset{tabsize=2}

\title{Сортировка камней.}
\author{Санду Р.А.}

\begin{document}
\maketitle

Утверждается, что ассимптотически оптимальным является следующий алгоритм.

Для сортировки будем использовать кучу. Добавим первые $k$ элементов в кучу. Мы могли бы продолжить и добавить следующий элемент, но из условия мы уже знаем, что этот $k$-й элемент не мог стоять на нулевой позиции, как и все последующие элементы. Значит, элемент, стоявший на нулевой позиции, находится в куче и является в ней минимальным. Извлечём его из кучи за $\log{k}$ и поставим на нулевую позицию, и только теперь добавим $k$-й элемент в кучу.
Таким образом, повтаряя этот процесс по аналогии и дальше, размер кучи никогда не превысит $k$, а следовательно каждый из $n$ элементов будет вставлен и извлечён за $\log{k}$, и ассимптотика этого алгоритма будет $O(n\log{k})$

Теперь дадим нижнюю оценку этой задаче. Рассмотрим перестановки, удовлетворяющих условию. Назовём их "хорошими" и обозначим их число за $m$. Любой алгоритм сортировки эквивалентен дереву сравнений. Пусть высота дерева сравнений равна $h$. Тогда всего возможных исходов у алгоритма столько же, сколько и листьев у этого дерева, т.е. $2^h$. Для каждой хорошей перестановки должен существовать исход алгоритма, сортирующий её. Иными словами, из мн-ва исходов должна существовать суръекция на мн-во хороших перестановок, т.е. кол-во исходов должно быть не меньше кол-ва хороших перестановок.

\[
	2^h \geq m
\]
\[
	h \geq \log_2{m}
\]

Дадим нижнюю оценку для m. Рассмотрим более узкий класс перестановок: такие перестановки, в которых массив разбит на $\frac{n}{k}$ идущих подряд непересекающихся подотрезков, элементы в каждом из которых переставлены друг с другом. Таких объектов будет ровно $k!^\frac{n}{k}$, а так как любая такая перестановка является хорошей, $m \geq k!^\frac{n}{k}$.

Соединив эти два рассуждения,

\[
	h \geq \log_2{k!^\frac{n}{k}}
\]
\[
	h \geq \frac{n}{k}\log_2{k!}
\]

Следующий переход был дан без доказательства на лекции.

\[
	h \geq \frac{n}{k}k\log_2{k}
\]
\[
	h \geq n\log_2{k}
\]

А так как время работы алгоритма пропорционально высоте дерева сравнений h, получаем нижнюю оценку $\Omega(n\log{k})$


\end{document}
