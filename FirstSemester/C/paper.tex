\documentclass[11pt]{article}

\usepackage[utf8]{inputenc}
\usepackage[russian]{babel}
\usepackage{listings}

\usepackage{geometry}
\geometry{a4paper}
\lstset{tabsize=2}

\title{Сравнение скорости работы std::cin и scanf.}
\author{Санду Р.А.}
\date{}

\begin{document}
\maketitle

В качестве данных для тестирования была выбрана последовательность, i-ый член которой является i-ым числом Фибоначчи по модулю $10^9+7$, делённым на $10^6$.
Для генерации был использован следующий код (\verb!generate.cpp!):
\lstinputlisting[language=c++]{generate.cpp}
\clearpage

Для проведения тестов была использованна следующая программа (\verb!clock.cpp!):
\lstinputlisting[language=c++]{clock.cpp}


А также для удобства был написан следующий скрипт, делающий всё и сразу \verb!test.sh!:
\lstinputlisting[language=bash]{test.sh}

С помощью программы-генератора были сгенерированны 4 файла с соответственно $10^4$, $10^5$, $10^6$ и $10^7$ числами, каждый из которых был по-отдельности считан программой-тестом, в результате чего были получены следующие данные:

\begin{center}
\begin{tabular}{ | c | c | c | }
	\hline
	Кол-во чисел & std::cin & scanf \\
	\hline
	$10^4$ & 46.875 & 31.25 \\
	$10^5$ & 437.5 & 125.0 \\
	$10^6$ & 3968.75 & 1421.88 \\
	$10^7$ & 41812.5 & 14703.1 \\
	\hline
	\hline
	\multicolumn{3}{|c|}{Время приведено в миллисекундах.} \\
	\hline
\end{tabular}
\end{center}

Из этих данных можно сделать очевидный вывод: std::cin работает намного медленнее, нежели scanf.
Но на самом деле, если использовать std:ifstream и fscanf, внезапно окажется, что потоки С++ работают быстрее (этот факт тоже был проверен).
Вывод номер два: не использовать freopen для считывания файлов.

\end{document}
