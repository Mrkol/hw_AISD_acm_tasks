\documentclass[11pt]{article}

\usepackage[utf8]{inputenc}
\usepackage[russian]{babel}
\usepackage{listings}
\usepackage{amsmath}
\usepackage{clrscode3e}


\usepackage{geometry}
\geometry{a4paper}
\lstset{tabsize=2}

\title{Тимсорт}
\author{Санду Р.А.}

\newcommand{\len}[0]{\operatorname{len}}
\newcommand{\hgt}[0]{\operatorname{hgt}}
\newcommand{\coins}[0]{\operatorname{coins}}

\begin{document}
\maketitle

Не сложно заметить, что первый этап алгоритма линеен: в худшем случае данные перемешаны настолько, что каждый ран (от англ. run) придётся добивать до минимальной длинны сортировкой вставками. Но тогда это займёт $O(\frac{n}{min\_run}min\_run^2) = O(n\cdot min\_run)$ времени, где $min\_run$ — константа порядка $2^5$

При добавлении рана $R$ в стек будем класть на него по $\coins R = \hgt R\cdot O(\len R)$ монет, где $\hgt R$ — индекс элемента в стеке, где $\hgt R = 0$ означает, что ран в самом низу стека, а $\len R$ — длинна рана. Пусть после добавления нового рана нарушился один из инвариантов. Обозначим за $S_i$ текущие раны в стеке, где $S_0$ — вершина стека, т.е. только что добавленный ран. Тогда инварианты имеют вид
\begin{equation}
	\len S_1 > \len S_0,
\end{equation}

\begin{equation}
	\len S_2 \geq \len S_1 + \len S_0.
\end{equation}
При слиянии будем класть монеты из сливаемых ранов на получившийся ран.
Также будем поддерживать всё время работы инвариант

\begin{equation}
	\coins S_i = \hgt R \cdot O(\len S_i).
\end{equation}

Первый случай: нарушился инвариант (1), $S_1 \leq S_0$.
Тогда сливатся будут $S_1$ и минимальный по длинне из $S_0$ и $S_2$, а значит нужно $\len S_1 + \min(\len S_0,\len S_2)$ монет на операцию слияния. Но так как в данном случае $\len S_1 < \len S_0$ и $\min(\len S_0, \len S_2) \leq \len S_0$, имеем 

\begin{equation}
	\len S_1 + \min(\len S_0,\len S_2) = O(\len S_0).
\end{equation}

Заметим также, что при любом из слияний $\hgt S_0$ гарантированно уменьшится на единицу, а значит мы сможем использовать $O(\len S_0)$ монет, не нарушая инвариант (3), и тогда в данном случае операция оплачена.


Второй случай: нарушился инвариант (2), $\len S_2 < \len S_1 + \len S_0$. 
Разберём первый подслучай: $\len S_2 < \len S_0$, сливать будем $S_2$ и $S_1$. Тогда получаем из нарушения $\len S_2 = O(\len S_1 + \len S_0)$, и прибавив к обоим частям $\len S_1$, получаем 
\begin{equation}
	\len S_1 + \len S_0 = O(\len S_0 + \len S_1).
\end{equation}
 А так как $\hgt S_0$ и $\hgt S_1$ уменьшатся, получается, что у нас есть те самые $O(\len S_0) + O(\len S_1)$ монет, чтобы оплатить операцию.

Разберём второй подслучай: $\len S_2 > \len S_0$, сливаем $S_1$ и $S_0$. Рассмотрим, что произойдёт после слияния, обозначив за $S_i \cup S_j$ слитые раны $S_i$ и $S_j$. Очевидно, что $\len (S_0 \cup S_1) > \len S_2$, а значит следующее слияние обязательно будет. Скажем, что в этом случае проверки инвариантов мы проводить не будем, а просто сразу начнём востанавливать его дальше. Проходить оно будет по первому случаю, и нам придётся на него потратить $O(\len (S_0 \cup S_1)) = O(\len S_0 + \len S_1)$ монет. Но тогда за счёт увеличения константы мы можем оплатить этими монетами и слияние $S_0$ и $S_1$.

Таким образом получается, что все операции оплачены. Учётная стоимость всего второго этапа получается

\begin{equation}
	\sum_{i=0}^{r}{l_i h_i} \leq \sum_{i=0}^{r} {l_i \bar{h}} \leq n \bar{h},
\end{equation}

где $l_i$ — длинна рана, добавленного i-м, $h_i$ — высота стека в момент добавления, $n$ — кол-во элементов в массиве, а $\bar{h}$ — верхняя оценка на высоту стека. 

Докажем, что $\bar{h} = O(\log{n})$. Из инварианта (2) получается, что в каждый момент времени $\len S_i$ ограничено снизу $F_i$ — i-м числом фибоначчи, где индексация $S_i$ идёт с низа стека. Но частичные суммы $\len S_i$ ограниченны $F_{i+2}$, а по формуле Бине $F_{i+2} = O(exp(i+2))$. Частичные суммы никогда не превысят $n$, а значит и нижняя грань тоже будет меньше или равна $n$. Тогда получаем, что $exp(\bar{h}) = O(n)$, а значит $\bar{h} = O(\log{n})$, что и требовалось доказать.

Далее, о том, в каком порядке востанавливается инвариант. Первичных проверок после добавления новго рана не больше, чем самих ранов, а значит оплатить их можно просто положив на константу монеток на каждый ран больше. Если инвариант сломался, то после его исправления достаточно посмотреть на 5 верхних ранов в стеке чтобы понять, сломан ли инвариант. Все эти проверки можно оплатить, положив на каждый ран на константу монеток больше, ведь после каждого исправления у нас есть $O(\len S_i)$ монет на то, чтобы оплатить востановление и проверку. Тогда каждую проверку можно оплатить константой монет на каждом из этих ранов. Таким образом, все проверки выполнения инварианта тоже оплачены.

Значит второй этап работает за $O(n\log{n})$. Из ограничение на кол-во слияний так же следует, что и третий этап работает за $O(n\log{n})$, и получаем, что суммарна весь алгоритм работает за $O(n\log{n})$.

\end{document}